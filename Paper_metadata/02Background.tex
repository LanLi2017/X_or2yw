% reproducibility issue
There are over than 70 percent of scientists failed to reproduce other people's work, and more than half of scientists cannot reproduce their own work \cite{baker2016reproducibility}. "Reproducibility crisis" often refers to the inconsistency in results across studies \cite{samsa2019guide}. Discussion about reproducibility should consider repeatability of data management and analysis with an identical answer, where data management is to reconstruct the original data and prepare it for further analysis. Data cleaning is one essential part of the data management, through which data errors would be addressed and fixed. An adopted taxonomy that covers all the error types is proposed; four types of data errors are categorized, one of which is named outliers classified as quantitative, the other includes duplicates, rule violations, and pattern violations are classified as qualitative ones \cite{abedjan2016detecting}. Different data cleaning tools will focus on different data errors, trying multiple approaches to decide on the estimated values that best support the hypothesis with rationale.

For preclinical research, an auditable record of original data is required for reproducible, including the rationale for data cleaning and retaining the analysis programs. As the dataset size is increasing, a formal software-based system for data cleaning and management is needed to create an analysis file, which is reproducible documentation of the data analysis \cite{samsa2019guide}. 
% Hybrid provenance is significant and necessary 
There are two forms of provenance, prospective and retrospective. \textit{prospective provenance} captures the procedure of computational processes, working as a recipe that must be followed to reproduce the data product or class of data products. \textit{retrospective provenance} captures the executed products during the computation, i.e., it's a detailed log of the computational task \cite{freire2008provenance}. Combining retrospective and prospective provenance can produce scientifically significant hybrid provenance representations of the computational histories of data produced during a script run \cite{zhang2017revealing}. This integration of provenance can both reveal the overall history of a script run and also the fine-grained history of particular script products. 

% hybrid provenance - cell level could fix reproducibility issue
Data cleaning is noteworthy in data science. Scientific reproducibility highly depends on a well-curated dataset, where it means specific data storage format specification, no lexico-graphic errors, or miscategorized entities, adequate representation of missing data inferred by the valid existing data \cite{nunez2020first}.  Data cleaning tools such as OpenRefine provides a recipe recording the \textit{operation history} in JSON format. This recipe is supposed to be reused and reproduces the data cleaning workflow, where it captures the incomplete computational steps as well as partial execution products, i.e., we named it incomplete hybrid provenance. The paper uncovers the history of individual cell contents, i.e., it provides a model that promotes cells as first-class citizens in the data cleaning workflow \cite{nunez2020first}. This cell-level retrospective provenance could answer questions about how a given value in a cell came to be, and what its individual operation history was. 



